Da nun die mathematischen Grundlagen erl�utert wurden, m�chte ich mit deren Anwendungen weiterfahren und im kommenden Teil nacheinander drei der vier Grundrechenarten besprechen. Ich werde jeweils erkl�ren, wie der Trick funktioniert, wenn n�tig eine algebraische Begr�ndung liefern und anschliessend ein Beispiel mit gezeigter Methode l�sen.
\subsection{Anwendungen bei der Addition}
Bei einer Addition werden zwei oder mehr Summanden zusammengez�hlt. Formal wird eine Addition so dargestellt:
\begin{equation*}  Summand_1 + Summand_2 + Summand_3 + \ldots + Summand_n = Summe \end{equation*} 
%%
%%
\subsubsection{Gruppen bilden}
Bei Additionen darf die Reihenfolge der Summanden frei gew�hlt werden. Dieser Ansatz beruht auf dem Kommutativgesetz (siehe Kapitel \ref{kommutativ}). Dies ist vor allem bei Kettenrechnungen ein sehr m�chtiges Werkzeug. Idealerweise werden die einzelnen Glieder einer Kettenrechnung so angeordnet, dass sie mit dem kleinstm�glichen Aufwand zusammengez�hlt werden k�nnen. 
\begin{equation*}\label{gruppen} 13 + 56 + 34 + 53 = \text{ ?} \end{equation*}
Versucht man Gruppen zu bilden, so wird auffallen, dass sich die Einerstellen der Zahlen 56 und 34 zusammen auf 10 erg�nzen, (56 + 34) zusammengez�hlt also eine Zehnerzahl (90) ergeben. Wendet man diesen Weg an, so kann die Rechnung wie folgt vereinfacht werden: \begin{equation*} (56 + 34) + 13 + 53 = 90 + 13 + 53 = 103 + 53 = 156 \end{equation*}
Jedoch k�nnte man auch sehen, dass die Einerstellen der Zahlen 13, 34, 53 sich auch auf 10 erg�nzen und die Rechnung wie folgt vereinfachen: \begin{equation*} (13 + 34 + 53) + 56 = 100 + 56  = 156 \end{equation*} Sowohl der erste als auch der zweite Weg sind korrekt. Man k�nnte auch im Kopf von links nach rechts immer eine Zahl zur n�chsten addieren, jedoch entsteht dabei eine Vielzahl an Zwischenschritten und Zwischenresultaten. Genau das gilt es beim Kopfrechnen zu vermeiden. Mit diesem Beispiel m�chte ich zeigen, dass es viele verschiedene Wege gibt, eine Aufgabe im Kopf zu rechnen. Es gibt dabei weder richtig noch falsch, vielmehr gibt es individuelle Rechenwege, die zum Resultat f�hren. Die Sch�lerInnen sollen ihren je eigenen Weg finden, der am logischsten erscheint.
%%
%%
\subsubsection{Von rechts nach links}
Da ich mich in meiner Arbeit mit dem Kopfrechnen befasse und meine ProbandInnen die Oberstufe besuchen, setze ich die Kenntnis der schriftlichen Addition voraus und werde sie hier nicht weiter erl�utern.\\ \\
Eine Methode bei der man exakt die gleichen Rechenschritte wie bei der schriftlichen Addition durchf�hrt, jedoch ohne sich die Zahlen untereinander aufzuschreiben, nennt sich \emph{von rechts nach links}. Man rechnet jeweils zuerst die Einer aller Summanden zusammen, beachtet die �berschl�ge und arbeitet sich �ber die Zehnerstelle zur Hunderterstelle vor. Ein Nachteil dieser Methode ist, dass viele Zwischenresultate zu merken sind. 
\begin{equation*}\label{vrnl} 438 + 237 = \text{ ?} \end{equation*}
Unter Anwendung der Methode \emph{von rechts nach links} z�hlt man die Einer, Zehner und Hunderter separat zusammen:
\begin{equation*} \text{Hunderter: } 4 + 2, \text{  Zehner: } 3 + 3, \text{  Einer: } 8 + 7 \end{equation*}
\begin{equation*} \text{Hunderter: } 6, \text{  Zehner: } 6, \text{  Einer: } 15 = 675\end{equation*} 
%%
%%
\subsubsection{Hin�berschieben} \label{hinuberschieben}
Die Summe ver�ndert sich nicht, wenn beim einen Summanden eine Zahl addiert wird, vorausgesetzt man subtrahiert beim anderen Summanden dieselbe Zahl wieder. Dieser Trick ist sehr hilfreich, um Zehner�berg�nge zu vermeiden. Der Zehner�bergang wird vermieden, indem die eine Zahl auf eine Zehnerzahl erg�nzt bzw. reduziert wird. Dadurch f�hrt man nicht einen Zehnerschritt im eigentlichen Sinne aus, sondern macht den Schritt zum vollen Zehner. Die Korrektheit dieses Tricks l�sst sich wie folgt verallgemeinern:
\begin{equation*} (a+ x) + (b - x) = a + b + ( x - x) = a + b \end{equation*}
Da wir die selbe Zahl ($x$) einmal dazugez�hlt und einmal abgezogen haben, ver�ndert dies das Resultat nicht.
\begin{equation*} 23 + 99 =\text{ ?} \end{equation*} In diesem Beispiel lohnt es sich, einen Einer von 23 zu 99 hin�berzuschieben, was uns zu einer L�sung bringt, die einfach im Kopf zu rechnen ist. \begin{equation*}23 + 99 = (23-1) + (99+1) = 22 + 100 = 122 \end{equation*}