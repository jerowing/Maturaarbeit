\section{Anwendungen}
Da nun die mathematischen Grundlagen erl�utert wurden, m�chte ich mit deren Anwendungen weiterfahren und im kommenden Teil nacheinander drei der vier Grundrechenarten besprechen. Ich werde jeweils erkl�ren, wie der Trick funktioniert, wenn n�tig eine algebraische Begr�ndung liefern und anschliessend ein Beispiel mit gezeigter Methode l�sen.
\subsection{Anwendungen bei der Addition}
Bei einer Addition werden zwei oder mehr Zahlen zusammengez�hlt. Formal wird eine Addition so dargestellt:
\begin{equation*}  Summand_1 + Summand_2 + Summand_3 + \ldots + Summand_n = Summe \end{equation*} 

\subsubsection{Von rechts nach links}
Da ich mich in meiner Arbeit mit dem Rechnen im Kopf befasse und meine Probanden die Oberstufe besuchen, setze ich die Kenntnis der schriftlichen Addition voraus und werde sie hier nicht weiter erl�utern.

Eine beliebte Methode, bei der man exakt die gleichen Rechenschritte wie bei der schriftlichen Addition durchf�hrt, jedoch ohne sich die Zahlen untereinander aufzuschreiben, nennt sich \emph{von rechts nach links}. Man rechnet jeweils zuerst die Einer aller Summanden zusammen, beachtet die �berschl�ge und arbeitet sich zur Hunderterstelle vor. Dies ist jedoch eine sehr brachiale Standardmethode. Klar f�hrt auch sie ans Ziel, doch muss man sich, und das sei der grosse Nachteil dieser Methode extrem viele Zwischenresultate merken. Ich m�chte niemanden der sich diese Methode zu eigen gemacht hat davon abhalten so zu rechnen, jedoch werde ich im folgenden einige weitere Tricks vorstellen, mit welchen man Additionen auf eine andere Weise l�sen kann.

\begin{equation*}\label{vrnl} 438 + 237 = \text{ ?} \end{equation*}
Unter Anwendung der Methode \emph{von rechts nach links} z�hlt man die Einer, Zehner und Hunderter separat zusammen:
\begin{equation*} \text{Hunderter: } 4 + 2, \text{  Zehner: } 3 + 3, \text{  Einer: } 8 + 7 \end{equation*}
\begin{equation*} \text{Hunderter: } 6, \text{  Zehner: } 6, \text{  Einer: } 15 = 675\end{equation*} 
\subsubsection{Gruppen Bilden}
Bei Additionen und Multiplikationen darf die Reihenfolge der Summanden bzw. Faktoren frei gew�hlt werden. Dies ist vor allem bei Kettenrechnungen ein sehr m�chtiges Werkzeug! Am besten ordnet man die einzelnen Glieder einer Kettenrechnung so an, dass man mit dem kleinstm�glichen Aufwand ans Ziel kommt. Diese Aussage geht auf das Kommutativgesetz (zu finden im Kapitel \ref{kommutativ}) zur�ck.
\begin{equation*}\label{gruppen} 13 + 56 + 34 + 53 = \text{ ?} \end{equation*}
Versucht man Gruppen zu bilden, so wird vielen Leuten auffallen, dass sich die Einerstellen der Zahlen 56 und 34 zusammen auf 10 erg�nzen, 56 + 34 also eine runde Zahl (90) ergeben. Wendet man diesen Weg an, so kann man die Rechnung wie folgt vereinfachen: \begin{equation*} (56 + 34) + 13 + 53 = 90 + 13 + 53 = 103 + 53 = 156 \end{equation*}
Jedoch k�nnte man auch sehen, dass die Einerstellen der Zahlen 13, 34, 53 sich auch auf 10 erg�nzen und die Rechnung wie folgt vereinfachen: \begin{equation*} (13 + 34 + 53) + 56 = 100 + 56  = 156 \end{equation*} Sowohl der erste als auch der zweite Weg sind korrekt, klar k�nnte man auch im Kopf von Links nach Rechts immer eine Zahl zur n�chsten addieren, jedoch entstehen dabei eine Vielzahl an Zwischenschritten / Zwischenresultaten und das wollen wir beim Kopfrechnen vermeiden. Mit diesem Beispiel m�chte ich zeigen, dass es viele verschiedene Wege gibt, eine Aufgabe im Kopf zu rechnen. Es gibt dabei weder richtig noch falsch, vielmehr sind es verschiedene Wege. Jeder Sch�ler soll also seinen eigenen Weg finden, der f�r ihn am logischsten erscheint.

\subsubsection{Hin�berschieben} \label{hinuberschieben}
Die Summe ver�ndert sich nicht, wenn beim einen Summanden eine Zahl addiert wird, wenn man beim anderen Summanden die selbe Zahl wieder subtrahiert. Diesen Trick ist sehr hilfreich um Zehner�berg�nge zu vermeiden. Man vermeidet den Zehner�bergang indem man die eine Zahl zu einer Zehnerzahl erg�nzt bzw. reduziert. Dadurch f�hrt man nicht einen Zehnerschritt im eigentlichen Sinne aus, sondern macht den Schritt zum vollen Zehner. Die Korrektheit dieses Tricks l�sst sich ganz einfach wie folgt verallgemeinern
\begin{equation*} (a+ x) + (b - x) = a + b + ( x - x) = a + b \end{equation*}
Ein ganz banales Beispiel hierf�r ist die Addition von 99. Was ergibt \begin{equation*} 23 + 99 =\text{ ?} \end{equation*} In diesem Beispiel lohnt es sich einen Einer von 23 zu 99 hin�berzuschieben. Also die Rechnung wie folgt umzuformen, was uns schlussendlich zu einer Rechnung bringt, die einfach im Kopf zu rechnen ist. \begin{equation*}23 + 99 = (23-1) + (99+1) = 22 + 100 = 122 \end{equation*}