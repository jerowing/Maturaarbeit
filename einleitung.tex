Kopfrechnen ist keine schwierige Sache, zumindest nach meiner Auffassung. Oftmals  besch�ftige ich mich mit der Frage, weshalb das, was mir so leicht f�llt, anderen so schwer f�llt. Kopfrechnen bedeutet f�r mich nicht, dass wenn man eine Rechnung wie $ 31 \cdot 29$ sieht, unmittelbar das Resultat 899 ausspuckt wie ein Taschenrechner. Vielmehr bedeutet Kopfrechnen f�r mich sich zu helfen wissen, wie man die Rechnung so vereinfachen kann, damit sie im Kopf l�sbar ist. Mit anderen Worten, will man Rechnungen im Kopf l�sen, so wie dies ein Taschenrechner tut, so wird es schnell komplex und man verliert die Lust am Kopfrechnen. Geht man jedoch anders an Rechnungen heran und probiert diese zuerst zu vereinfachen und anschliessend eine einfachere Rechnung im Kopf zu l�sen, so wird man erfolgreich sein und �fters den Kopf anstelle des Taschenrechners brauchen. \textcolor{red}{In meiner Arbeit m�chte ich eine Gruppe von Oberstufensch�lern dazu motivieren im Alltag nicht sofort zum im Smartphone integrierten Taschenrechner zu greifen, sondern solche Rechnungen im Kopf zu l�sen} Um dies zu erreichen, m�chte ich den Sch�lern einige dieser Tricks weitergeben. Daf�r habe ich eine Doppelllektion zu Verf�gung. Jedoch sich der gew�nschte Effekt nicht nach dieser Doppellektion einstellen, vielmehr m�ssen die neu erlernten Methoden angewendet und bestenfalls von den Sch�lern selbst weiterentwickelt und verfeinert werden. Deshalb l�sen die Sch�ler nach dieser Doppellektion �bern einen Monat zu Beginn jeder Mathestunde 5 Minuten Aufgaben. Dadurch wird meine Arbeit in drei Teile aufgeteilt. Die Planung der Doppellektion, was will ich vermitteln, wie will ich es vermitteln, welche Probleme sind dabei aufgetreten. Im zweiten Teil werde ich dokumentieren, wie die Durchf�hrung abgelaufen ist, was hat so geklappt, wie ich es mir w�nschte, was nicht. Im letzten Teil bereite ich das ganze nochmals auf und schaue auf mein Projekt zur�ck. Habe ich mein Ziel erreicht?, was m�sste ich f�r eine weitere Durchf�hrung verbessern? Dieser Teil basiere ich auf der R�ckmeldung der Lehrperson (Herr Jud) und dem Verlauf der Aufgaben die die Sch�ler zu Beginn jeder Lektion l�sten mussten.