\section{Inhalte der Lektion}
Das Ziel meiner Lektion war den Sch�lern die Freude am Kopfrechnen zu vermitteln, die ich besitze. Dies wollte ich erreichen, indem ich ihnen einige Tricks beibringen wollte, mit denen sich Rechnungen so vereinfachen lassen und sich so im Kopf l�sbar sind. Im folgenden Kapitel geht es darum, was ich vermitteln m�chte. Es werden die mathematischen Grundlagen und wie die Tricks funktionieren erkl�rt.
\subsection{�bergeordnetes Ziel}
Bei der Planung der Lektion waren mit vielerlei Dinge sehr wichtig:
\begin{itemize}
	\item Verst�ndlich f�r alle Sch�ler
	\item verschiedene Werkzeuge f�r die Sch�ler, damit sie selbst kreativ werden k�nnen
	\item das Verst�ndnis des mathematischen Grundgedankens
	\item Alle Sch�ler sollen Freude am Kopfrechnen und der Mathe erhalten
	\item Sch�ler sollen daraus profitieren
\end{itemize}

Ich m�chte den Sch�lern diverse Tricks beibringen, die ihnen helfen Rechnungen zu vereinfachen. Diese Tricks sind im wesentlichen algebraische Umformungen und mathematisch korrekt! Damit kein Durcheinander entsteht, bespreche und erl�utere ich im kommenden Teil nacheinander die vier Grundrechenarten: den Hintergrund und die Tricks, welche f�r die jeweilige Rechenart angewendet werden k�nnen bez�glich der jeweiligen Rechenart. \\ \\ Die meisten Tricks bauen auf den 3 Grundgesetze der Mathematik auf. Die Sch�ler haben sie im Unterricht bereits kennengelernt, deshalb werde ich sie hier nur kurz anschneiden und nicht gr�ndlich ausf�hren.
\subsection{die Grundgesetze der Mathematik}
\subsubsection{Assoziativgesetz}\label{assoziativ} 
\begin{equation} (a + b) + c = a + (b + c) \end{equation}
In Worten ausgedr�ckt heisst das, dass die Reihenfolge der Ausf�hrung keine Rolle spielt. Es spielt also keine Rolle ob ich zuerst a und b zusammenz�hle und dann c addiere oder zuerst b und c addiere und dann a dazuz�hle. Das Assoziativgesetz ist g�ltig f�r die Addition und die Multiplikation, jedoch nicht f�r die Subtraktion und Division.


\subsubsection{Distributivgesetz} \label{distributiv}
\begin{equation} a \cdot (b + c) = a\cdot b + a \cdot c \end{equation}
Das Distributivgesetz besagt, dass bei der Multiplikation eines Faktors mit einer Summe (oder auch einer Differenz) die Multiplikation in zwei Teilschritte aufgeteilt werden darf, indem man die beiden Summanden (in der Allgemeinen Formel b und c) einzeln mit dem Faktor a multipliziert und aus diesen zwei Teilprodukten ($a\cdot c$ , $b \cdot c$) die Summe bildet.
Ausformuliert bedeutet das Distributivgesetz, dass bei einer Multiplikation die Faktoren in Summe aufgeteilt werden und nach obigem Schema weitergerechnet werden kann, das Distributivgesetz gilt f�r den allgemeinen Fall einer Multiplikation, wobei die Faktoren sowohl in Summen als auch in eine Differenzen aufgeteilt werden d�rfen. \\
\textcolor{red}{Das Distributivgesetz, kann auch angewendet werden, wenn beide Faktoren als Summe bzw. Differenz vorliegen:}

\subsubsection{Kommutativgesetz}
\begin{equation}  a + b =  b + a \label{kommutativ} \end{equation}
Hier wird verdeutlicht, dass die Anordnung der einzelnen Summanden das Resultat nicht beeinflusst. Bei der Addition und der Subtraktion darf die Reihenfolge, in der man die einzelnen Summanden zusammenz�hlt bzw. die Faktoren multipliziert frei gew�hlt werden. Konkret erlaubt uns dieses Gesetz die einzelnen Summanden oder Faktoren so zu gruppieren, dass es uns leichter f�llt sie zusammenzuz�hlen. Mehr dazu auf Seite \ref{gruppen}.