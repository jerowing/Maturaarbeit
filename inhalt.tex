\section{(KICKEN) Inhalte der Lektion}
Das Ziel meiner Lektion war den Sch�lern die Freude am Kopfrechnen zu vermitteln, die ich besitze. Dies wollte ich erreichen, indem ich ihnen einige Tricks beibringen wollte, mit denen sich Rechnungen so vereinfachen lassen und sich so im Kopf l�sbar sind. Im folgenden Kapitel geht es darum, was ich vermitteln m�chte. Es werden die mathematischen Grundlagen und wie die Tricks funktionieren erkl�rt.
\subsection{�bergeordnetes Ziel}
Bei der Planung der Lektion waren mit vielerlei Dinge sehr wichtig. Einerseits wollte ich den Sch�lern m�glichst viele verschiedenen Werkzeuge auf den Weg geben, jedoch so, dass sie von allen Sch�lern verstanden werden. Weiter wollte ich, das meine Freude am Kopfrechnen auf einige Sch�ler �berspringt und dass hoffentlich einige von ihnen den mathematischen Grundgedanken mit auf den Weg nehmen. Am wichtigsten war mir jedoch, dass alle Sch�ler aus dem Projekt profitieren konnten.  \\ \\
Um dies zu erreichen, m�chte den Sch�lern diverse Tricks beibringen, die ihnen helfen Rechnungen zu vereinfachen. Die Tricks sind im wesentlichen algebraische Umformungen und mathematisch korrekt! Die meisten Tricks greifen auf die drei mathematischen Grundgesetze zur�ck, da sie die Sch�ler bereits kennengelernt haben, werde ich sie im kommenden Teil nur kurz erl�utern.


\section{Mathematische Grundlagen}
Unter einer Vielzahl mathematischer Rechenregeln, gibt es drei besonders grundlegende. Diese heissen: Kommutativgesetz, Assoziativgesetz und Distributivgesetz. Diese drei sind besonders fundamental, da beinahe die ganze Mathematik auf ihnen aufbaut. Diese Drei Gesetze gelten f�r alle Zahlen, die in der Menge der rationalen Zahlen enthalten sind.

\subsection{Kommutativgesetz}
\begin{equation*}  a + b =  b + a \label{kommutativ} \end{equation*}
\begin{equation*} a \cdot b = b \cdot a \end{equation*}
Hier wird verdeutlicht, dass die Anordnung der einzelnen Summanden das Resultat nicht beeinflusst. Bei der Addition und der Subtraktion darf die Reihenfolge, in der man die einzelnen Summanden zusammenz�hlt bzw. die Faktoren multipliziert frei gew�hlt werden. Konkret erlaubt uns dieses Gesetz die einzelnen Summanden oder Faktoren so zu gruppieren, dass es uns leichter f�llt sie zusammenzuz�hlen. Mehr dazu in Kapitel \ref{gruppen}.

\subsection{Assoziativgesetz}\label{assoziativ} 
\begin{equation*} (a + b) + c = a + (b + c) \end{equation*}
\begin{equation*} (a \cdot b) \cdot c = a \cdot (b \cdot c) \end{equation*} 
In Worten ausgedr�ckt heisst das, dass die Reihenfolge der Ausf�hrung keine Rolle spielt. Es spielt also keine Rolle ob ich zuerst a und b zusammenz�hle und dann c addiere oder zuerst b und c addiere und dann a dazuz�hle. Das Assoziativgesetz ist g�ltig f�r die Addition und die Multiplikation, jedoch nicht f�r die Subtraktion und Division.


\subsection{Distributivgesetz} \label{distributiv}
\begin{equation*} a \cdot (b + c) = a\cdot b + a \cdot c \end{equation*}
Das Distributivgesetz besagt, dass bei der Multiplikation eines Faktors mit einer Summe (oder auch einer Differenz) die Multiplikation in zwei Teilschritte aufgeteilt werden darf, indem man die beiden Summanden (in der Allgemeinen Formel b und c) einzeln mit dem Faktor a multipliziert und aus diesen zwei Teilprodukten ($a\cdot c$ , $b \cdot c$) die Summe bildet. \\ 
%Ausformuliert bedeutet das Distributivgesetz, dass bei einer Multiplikation die Faktoren in Summe aufgeteilt werden und nach obigem Schema weitergerechnet werden kann, das Distributivgesetz gilt f�r den allgemeinen Fall einer Multiplikation, wobei die Faktoren sowohl in Summen als auch in eine Differenzen aufgeteilt werden d�rfen. Denn eine Subtraktion kann auch als Summe mit einem negativen Summanden dargestellt werden. \begin{equation*} 10 - 5 = 10 + (-5)\end{equation*}\\
Das Distributivgesetz, kann auch angewendet werden, wenn beide Faktoren als Summe bzw. Differenz vorliegen:
\begin{equation*} (a + b) \cdot (c + d) = \text{ ?} \end{equation*}
\begin{equation*} a\cdot (c + d) + b \cdot (c + d) = ac + ad + bc + bd\end{equation*}

