\section{Mathematische Grundlagen}
Unter einer Vielzahl mathematischer Rechenregeln gibt es drei grundlegende. Diese heissen: Kommutativgesetz, Assoziativgesetz und Distributivgesetz. Diese drei sind fundamental, da beinahe die ganze Mathematik auf ihnen aufbaut. Diese drei Gesetze gelten f�r alle Zahlen, die in der Menge der reellen Zahlen enthalten sind.
%%
\subsection{Kommutativgesetz}
\begin{equation*}  a + b =  b + a \label{kommutativ} \end{equation*}
\begin{equation*} a \cdot b = b \cdot a \end{equation*}
Hier wird verdeutlicht, dass die Anordnung der einzelnen Summanden bzw. Faktoren das Resultat nicht beeinflusst. Bei der Addition und der Multiplikation darf die Reihenfolge, in der man die einzelnen Summanden zusammenz�hlt bzw. die Faktoren multipliziert, frei gew�hlt werden. Dieses Gesetz erlaubt uns, die einzelnen Summanden oder Faktoren so zu anzuordnen, dass es uns leichter f�llt, sie zusammenzuz�hlen. 
%%
\subsection{Assoziativgesetz}\label{assoziativ} 
\begin{equation*} (a + b) + c = a + (b + c) \end{equation*}
\begin{equation*} (a \cdot b) \cdot c = a \cdot (b \cdot c) \end{equation*} 
In Worten ausgedr�ckt heisst das, dass die Reihenfolge der Ausf�hrung keine Rolle spielt. Es spielt also keine Rolle, ob ich zuerst $a$ und $b$ zusammenz�hle und dann $c$ addiere oder zuerst $b$ und $c$ addiere und dann $a$ dazuz�hle. Das gleiche Prinzip gilt auch f�r die Multiplikation. Das Assoziativgesetz ist g�ltig f�r die Addition und die Multiplikation, jedoch nicht f�r die Subtraktion und Division.
%%
\subsection{Distributivgesetz} \label{distributiv}
\begin{equation*} a \cdot (b + c) = a\cdot b + a \cdot c \end{equation*}
\begin{equation*} a \cdot (b - c) = a\cdot b - a \cdot c \end{equation*}
Das Distributivgesetz besagt, dass bei der Multiplikation eines Faktors mit einer Summe (oder auch einer Differenz) die Multiplikation in zwei Teilschritte aufgeteilt werden darf, indem man die beiden Summanden (in der allgemeinen Formel $b$ und $c$) einzeln mit dem Faktor $a$ multipliziert und aus diesen zwei Teilprodukten ($a\cdot c$ , $b \cdot c$) die Summe (bzw. Differenz) bildet.
Das Distributivgesetz, kann auch angewendet werden, wenn beide Faktoren als Summe bzw. Differenz vorliegen:
\begin{equation*} (a + b) \cdot (c + d) = \text{ ?} \end{equation*}
\begin{equation*} a\cdot (c + d) + b \cdot (c + d) = ac + ad + bc + bd\end{equation*}