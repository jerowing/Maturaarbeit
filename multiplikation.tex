\subsection{Anwendungen bei der Multiplikation}
Die Multiplikation wird allgemein wie folgt dargestellt und kann aus beliebig vielen Faktoren zusammengesetzt sein.
\begin{equation*} Faktor_1 \cdot Faktor_2 \cdot Faktor_3 \cdot \ldots \cdot Faktor_n = Produkt \end{equation*}

\subsubsection{\label{summenbilden} Summen Bilden}
Wandelt man einen Faktor in eine Summe um, so entstehen zwei Teilrechnungen. Es macht vor allem dann Sinn, diesen Trick anzuwenden, wenn man das eine Teilresultat bereits kennt. Beispiele solcher Teilresultate k�nnen Zehnerzahlen, Quadratzahlen aber auch das kleine Einmaleins sein. Vor allem bei der Multiplikation mit Zahlen die nahe an einem Zehner liegen ist dies ein Mittel um sehr schnell an das Resultat zu gelangen. Dieser Trick ist eine Anwendung des Distributivgesetzes (siehe Kapitel \ref{distributiv}). Vielfach wird das Distributivgesetz unbewusst angewandt. Rechnungen wie $15 \cdot 7$ werden vielfach intuitiv als $10\cdot 7 + 5 \cdot 7$ gel�st, was genau der Anwendung dieses Tricks entspricht. Jedoch m�chte ich die M�chtigkeit dieses Trickes am folgenden Beispiel zeigen:
\begin{equation*} 13 \cdot 12 = \text{ ?} \end{equation*}
Wir wissen, dass 12 im Quadrat 144 ergibt, deshalb k�nnen wir 13 zu $(12 + 1)$ umformen.
\begin{equation*} (12 + 1) \cdot 12 = 12^2 + 12 = 144 + 12 = 156\end{equation*} ohne grossen Rechenaufwand sind wir zum Resultat gelangt.

\subsubsection{\label{doppelteSumme}Doppelte Summe}
Die Methode "`doppelte Summe"' ist im Grunde genommen nichts anderes als die Weiterf�hrung der Methode "`Summen bilden"'. Statt einen Faktor in eine Summe umzuwandeln, bildet man in diesem Fall aus beiden Faktoren je eine Summe bildet. Allgemein sieht eine Multiplikation zweier Summen so aus:
\begin{equation*} (a+b) \cdot (c+d) = ac + ad + bc + bd \end{equation*}  
Diese Formel mag f�r alle Menschen, die mit Algebra auf Kriegsfuss stehen unsch�n erscheinen. Keine Angst, die Formel kann und wird noch vereinfacht werden, unter Einf�hrung der Nebenbedingung, dass in beiden Summen der gleiche Summand vorkommen soll. Wir gehen davon aus, dass die Summen so gew�hlt wurden, dass die Variable a und c den gleichen Wert haben und deshalb beide mit der Variabel a dargestellt werden k�nnen:
\begin{equation*} (a+b) \cdot (a+d) = a^2 + ab + ad + bd\end{equation*}
Unter Anwendung des Distributivgesetzes (kap. \ref{distributiv}) kann der Term wie folgt vereinfacht werden:
\begin{equation*} a^2 + ab + ad + bd = a^2 + a\cdot(b+d) + bd \end{equation*}
Somit wurde die Multiplikation von zwei Faktoren umgewandelt in die Addition von einer Quadratzahl und zwei Produkten.
Damit diese Endformel auch wirklich f�r jedermann im Kopf l�sbar ist, sollten die Zahlen geschickt gew�hlt werden. Besonders einfach wird die Rechnung, wenn die Variable a so gew�hlt wird, dass sie eine ganze Zehnerzahl ist. Dass man dank diesem Kniff eine schwierige Multiplikation, im Kopf l�sen kann, m�chte ich mit folgendem Beispiel untermauern.
\begin{equation*} 23 \cdot 16 =\text{ ?} \end{equation*} 
Als erstes wandeln wir beide Faktoren in eine Summe um und achten darauf, dass in beiden Summen der gleiche Summand vorkommt. Hat man zwei Zahlen in der N�he des gleichen Zehners, in unserem Beispiel 20, so macht es Sinn die Summen mit diesem Zehner zu bilden. Streng genommen ist die Umwandlung von $16$ zu $(20 - 4)$ keine Addition sondern eine Subtraktion, auch das ist erlaubt. \begin{equation*} (20+3) \cdot (20-4) =\text{ ?} \end{equation*}
\begin{equation*}  (20+3) \cdot (20 - 4) = 20^2 + (3 - 4) \cdot 20 - 3 \cdot 4  = \text{ ?} \end{equation*}
Aus vier Zwischenresultaten wurden drei und von diesen dreien ist eines ($a^2$) eine Quadratzahl, sollte also den Sch�lern bekannt sein. Also bleiben noch zwei Rechenschritte �brig: $a\cdot(b+d)$ Hier ist es wichtig, dass zuerst die Klammer ausgerechnet und erst anschliessend die Multiplikation durchgef�hrt wird, um den Aufwand m�glichst gering zu halten. Zum Schluss muss einzig noch die Rechnung $b \cdot d$ wirklich ausgerechnet werden. Im normalen Fall sollte dies jedoch mit dem kleinen Einmaleins m�glich sein. Also haben wir zum Schluss die folgende Rechnung:
\begin{equation*} 23 \cdot 16 = 400 - 20 - 12 = 400 - 32 = 368\end{equation*}

\subsubsection*{Ein Spezialfall}
Die dritte binomische Formel sieht wie folgt aus:
\begin{equation*} (a + b) \cdot (a - b) = a^2 - b^2  \end{equation*}
Liegen die Zahlen g�nstig, so kann man die dritte binomische Formel anwenden. Die dritte binomische Formel kann dann angewendet werden wenn zwei unterschiedliche Zahlen den gleichen Abstand zu einer dritten Zahl haben. Einige Beispiele um diesen Spezialfall zu verdeutlichen:
\begin{equation*} 22 \cdot 18 = (20 + 2) \cdot (20 - 2) = 20^2 - 2^2 = 400 - 4 = 396 \end{equation*}
\begin{equation*} 14 \cdot 16 = (15 - 1) \cdot (15 + 1) = 15^2 - 1^2 = 225 - 1 = 224 \end{equation*}

\subsubsection{Faktoren Aufteilen}
Eine Multiplikation in mehrere Schritte aufgeteilt werden. Beispielsweise darf eine Multiplikation mit 4 als eine Multiplikation mit 2 und nochmals mit 2 angesehen werden. Oder eine Multiplikation mit 5 darf als eine Multiplikation mit 10 und eine anschliessende Division mit 2 (�quivalent zu der Multiplikation mit $\frac{10}{2}$) Grund daf�r ist das Assoziativgesetz.
N�here Erl�uterungen m�chte ich am folgenden Beispiel durchf�hren.
 \begin{equation*} 4 \cdot 27 \end{equation*}
Betrachtet man die formale Schreibweise des multiplikativen Assoziativgesetzes (Formel \ref{assoziativ} auf Seite \pageref{assoziativ}) so kann man die Rechnung auf die folgenden Arten anschauen: Man interpretiert die Rechnung so, dass der Faktor 4 als $(a \cdot b)$ angesehen wird, also s�he die Rechnung so aus: 
\begin{equation*} (4) \cdot 27 = \text{ ?} \end{equation*} 
Wie das Assoziativgesetz besagt, k�nnen in der Klammer mehrere Faktoren stehen. Damit sich das Ergebnis der Rechnung nicht ver�ndert muss jedoch das Produkt in der Klammer 4 ergeben. Es bieten sich die folgenden Rechnungen an: $ 1 \cdot 4 $ oder $2 \cdot 2$ (nat�rlich g�be es noch viel mehr Multiplikationen mit dem Ergebnis 4. Diese Rechnungen w�rden jedoch negative Zahlen und oder Br�che beinhalten und sowohl negative Zahlen als auch Br�che sind schwieriger zu handhaben als positive, ganze Zahlen.)
Weil mit der Umformung $1 \cdot 4$ keine wirkliche Ver�nderung stattfindet, fahren wir mit der zweiten M�glichkeit ($4 = 2 \cdot 2)$ fort. Also schreiben wir die Rechnung so: 
\begin{equation*} (2 \cdot 2) \cdot 27 = \text{ ?} \end{equation*} 
Da die Multiplikation auch kommutativ ist, d�rfen die Faktoren in eine beliebige Reihenfolge gebracht werden. Da die Multiplikation mit 2 einfacher ist als die Multiplikation mit 27 nehmen wir die 27 an den Anfang und multiplizieren diese zweimal mit der Zwei.
\begin{equation*} (27 \cdot 2) \cdot 2 = (54) \cdot 2 = 108 \end{equation*} Es ist jedoch auch erlaubt, die 27 in Faktoren zu zerlegen. Hier gibt es wiederum zwei M�glichkeiten: $1 \cdot 27$ und $3 \cdot 9$. Wie im vorherigen Beispiel wird mit der Umformung von 27 zu $27 \cdot 1$ keine Vereinfachung herbeigef�hrt. Deshalb fahren wir weiter mit der Umformung von 27 zu $3 \cdot 9$ fort und erhalten die folgende Rechnung:
\begin{equation*} 4 \cdot (27) = 4 \cdot (3 \cdot 9) = 4 \cdot 3 \cdot 9 = 12 \cdot 9 \end{equation*}
So wurde aus der Rechnung $4 \cdot 27$ eine v�llig andere Rechnung $ 9 \cdot 12$. Wie man das einfach ausrechnen kann, wird im n�chsten Kapitel erkl�rt.

