\subsection{Multiplikation}
\subsubsection*{Was ist eine Multiplikation}

\subsubsection{Faktoren Aufteilen}
Da die Multiplikation assoziativ ist, darf eine Multiplikation in mehrere Schritte aufgeteilt werden. Beispielsweise darf eine Multiplikation mit 4 als eine Multiplikation mit zwei und nochmals mit zwei angesehen werden. Oder eine Multiplikation mit 5 darf als eine Multiplikation mit 10 und eine anschliessende Division mit 2 (�quivalent zu der Multiplikation mit $\frac{1}{2}$
\begin{equation} \label{faktorenaufteilen} (a \cdot b) \cdot c =  a \cdot (b \cdot c) = a \cdot b \cdot c \end{equation}
N�here Erl�uterungen m�chte ich am Beispiel der Rechnung $4 \cdot 27$ durchf�hren.
man kann die Rechnung wie folgt ansehen: \begin{equation*} 4 \cdot 27 \end{equation*}
Betrachtet man die formale Schreibweise des multiplikativen Assoziativgesetzes (Formel \ref{faktorenaufteilen} auf Seite \pageref{faktorenaufteilen}) so kann man die Rechnung auf die folgenden Arten anschauen: \begin{itemize}
\item Man interpretiert die Rechnung so, dass der Faktor 4 als $(a \cdot b)$ angesehen wird also s�he die Rechnung so aus: \begin{equation*} (4) \cdot 27 = ? \end{equation*} Wie in Formel \ref{faktorenaufteilen} zu sehen ist, k�nnen in der Klammer mehrere Faktoren stehen. Damit sich das Ergebnis der Rechnung nicht ver�ndert muss jedoch das Produkt in der Klammer 4 ergeben. Es bieten sich die folgenden Rechnungen an: $ 1 \cdot 4 $ oder $2 \cdot 2$ (nat�rlich g�be es noch viel mehr Multiplikationen mit dem Ergebnis 4. Diese Rechnungen w�rden jedoch negative Zahlen und oder Br�che beinhalten und sowohl negative Zahlen als auch Br�che sind schwieriger zu handhaben als positive, ganze Zahlen.)
Da mit der Umformung $1 \cdot 4$ lediglich \textcolor{red}{Anschauungskosmetik} ist, da die Multiplikation mit 1 vernachl�ssigt werden kann, fahren wir mit der zweiten M�glichkeit ($4 = 2 \cdot 2)$ Also schreiben wir die Rechnung so: \begin{equation*} (2 \cdot 2) \cdot 27 = ?? \end{equation*} Da die Multiplikation kommutativ ist, d�rfen die Faktoren in eine beliebige Reihenfolge gebracht werden. Da die Multiplikation mit 2 einfacher ist als die Multiplikation mit 27 nehmen wir die 27 an den Anfang und multiplizieren diese zweimal mit der Zwei.
\begin{equation*} (27 \cdot 2) \cdot 2 = (54) \cdot 2 = 108 \end{equation*}

\item Es ist jedoch auch erlaubt, die 27 in Faktoren zu zerlegen. Hier gibt es wiederum zwei M�glichkeiten: $1 \cdot 27$ und $3 \cdot 9$. Wie im vorherigen Beispiel wird mit der Umformung von 27 zu $27 \cdot 1$ keine Vereinfachung herbeigef�hrt. Deshalb fahren wir weiter mit der Umformung von 27 zu $3 \cdot 9$ fort und erhalten die folgende Rechnung:
\begin{equation*} 4 \cdot (27) = 4 \cdot (3 \cdot 9) = 4 \cdot 3 \cdot 9 = 12 \cdot 9 \end{equation*}
So wurde aus der Rechnung $4 \cdot 27$ eine v�llig andere Rechnung $ 9 \cdot 12$. Wie man das einfach ausrechnen kann, wird im n�chsten Kapitel erkl�rt.
\end{itemize}

\subsubsection*{H����?}
Das Resultat (\sout{Produkt}) einer Multiplikation ver�ndert sich nicht, wenn ein Faktor mit einer Zahl multipliziert wird und ein anderer Faktor durch die selbe Zahl geteilt wird. Dankbar f�r diesen Fall sind alle Zahlen die Vielfache von zwei sind. Beispielsweise die Multiplikation mit 8: \begin{equation*} 125 \cdot 8  = 250 \cdot 4 = 500 \cdot 2 = 1000 \end{equation*} Die Multiplikation mit 8 wurde substituiert mit drei Multiplikationen mit Zwei.

\subsubsection{Summen Bilden}
Um diesen Trick anzuwenden macht man sich das Distributivgesetz zu nutze indem man sich aus einem Faktor eine Summe bildet. Es entstehen zwei Teilrechnungen, deshalb macht es vor allem dann Sinn, diesen Trick anzuwenden, wenn man das eine Teilresultat bereits kennt. Vor allem bei der Multiplikation mit Zahlen die nahe an einem Zehner liegen ist dies ein Mittel um sehr schnell an das Resultat zu gelangen. Im vorangehenden Kapitel tauchte die Rechnung $9 \cdot 12$ auf. es gibt wiederum zwei F�lle wie man diese Rechnung l�sen kann mit Hilfe des Distributivgesetzes:

\begin{itemize}
\item Man macht aus der neun eine Summe: $ 9 = (10 + (-1)) = (10 - 1)$ und multipliziert diese mit 12:
\begin{equation*} (10 - 1) \cdot 12 = 10 \cdot 12 - 1 \cdot 12 = 120 - 12 = 108 \end{equation*}
Die Multiplikation mit 10 ist keine schwierige Sache, lediglich eine null muss am Ende der Zahl angeh�ngt werden und die Multiplikation mit 1 er�brigt sich auch. Also erh�lt man die Zwischenresultate ohne grosse Rechnereien. Die Subtraktion am Schluss d�rfte den meisten leichter von der Hand gehen als die Urspr�ngliche Rechnung im Kopf durchzuf�hren.
\item Man verwandelt die 12 in die Summe $12 = (10 + 2)$ und rechnet auf die gleiche Weise wie eben mit Hilfe des Distributivgesetzes weiter:
\begin{equation*} 9 \cdot (10 + 2) = 90 + 18 = 108 \end{equation*} Hier tauchen wiederum zwei Multiplikationen auf, deren Produkte zusammengez�hlt werden m�ssen. Die Multiplikation $ 9 \cdot 10$ ist ein Kinderspiel, und die Rechnung $2 \cdot 9$ geh�rt zum kleinen Einmaleins und erledigt sich auch ohne grosse Schwierigkeiten.
\end{itemize}

\subsubsection{Fingertricks}
Die Methode Fingertricks ist eine Veranschaulichung der Methode "Summen Bilden". Sie ist anwendbar, wenn die beiden zu multiplizierenden \textcolor{red}{Zahlen in der N�he des gleichen Zehners liegen und der Abstand zu diesem nicht gr�sser als 5 ist.} Zuerst werden jede Zahl mit einer Hand dargestellt. F�r jeden Einer der zum Zehner dazugez�hlt werden muss, wird ein Finger nach oben gestreckt. Nicht gebrauchte Finger bleiben an der Hand angelegt und spielen keine Rolle. F�r jeden Einer der vom Zehner abgez�hlt werden muss, wird ein Finger nach unten abgespreizt. F�r das Beispiel $ 23 \cdot 16$ sieht das wie folgt aus: \textcolor{red}{BILD Einf�gen}
Was mit diesem \textcolor{green}{Schritt auf \textcolor{red}{ visuell-motorischer Ebene} durchgef�hrt wurde} ist nichts anderes als dass die zwei Zahlen 16 und 23 als Summen (20 + (-4)) und (20 + 3) dargestellt.

\subsubsection*{ohne Finger}


\subsubsection{Dritte Binomische Formel}
Die dritte binomische Formel sieht wie folgt aus:
\begin{equation*} (a + b) \cdot (a - b) = a^2 - b^2  \end{equation*}
Diese Formel kann man sich zu nutze Machen, sobald man zwei Zahlen haben, die nahe beieinander liegen, bevorzugt mit einem Zehner in der Mitte.  Beispiel: $22 \cdot 18 = (20 + 2) \cdot (20 - 2) = 400 - 4 = 496$