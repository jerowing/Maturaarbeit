\subsection{Multiplikation}
%\subsubsection*{Was ist eine Multiplikation}

\subsubsection{Faktoren Aufteilen}
Da die Multiplikation assoziativ ist, darf eine Multiplikation in mehrere Schritte aufgeteilt werden. Beispielsweise darf eine Multiplikation mit 4 als eine Multiplikation mit zwei und nochmals mit zwei angesehen werden. Oder eine Multiplikation mit 5 darf als eine Multiplikation mit 10 und eine anschliessende Division mit 2 (�quivalent zu der Multiplikation mit $\frac{1}{2}$
\begin{equation} \label{faktorenaufteilen} (a \cdot b) \cdot c =  a \cdot (b \cdot c) = a \cdot b \cdot c \end{equation}
N�here Erl�uterungen m�chte ich am Beispiel der Rechnung $4 \cdot 27$ durchf�hren.
man kann die Rechnung wie folgt ansehen: \begin{equation*} 4 \cdot 27 \end{equation*}
Betrachtet man die formale Schreibweise des multiplikativen Assoziativgesetzes (Formel \ref{faktorenaufteilen} auf Seite \pageref{faktorenaufteilen}) so kann man die Rechnung auf die folgenden Arten anschauen: \begin{itemize}
\item Man interpretiert die Rechnung so, dass der Faktor 4 als $(a \cdot b)$ angesehen wird also s�he die Rechnung so aus: \begin{equation*} (4) \cdot 27 = ? \end{equation*} Wie in Formel \ref{faktorenaufteilen} zu sehen ist, k�nnen in der Klammer mehrere Faktoren stehen. Damit sich das Ergebnis der Rechnung nicht ver�ndert muss jedoch das Produkt in der Klammer 4 ergeben. Es bieten sich die folgenden Rechnungen an: $ 1 \cdot 4 $ oder $2 \cdot 2$ (nat�rlich g�be es noch viel mehr Multiplikationen mit dem Ergebnis 4. Diese Rechnungen w�rden jedoch negative Zahlen und oder Br�che beinhalten und sowohl negative Zahlen als auch Br�che sind schwieriger zu handhaben als positive, ganze Zahlen.)
Da mit der Umformung $1 \cdot 4$ lediglich \textcolor{red}{Anschauungskosmetik} ist, da die Multiplikation mit 1 vernachl�ssigt werden kann, fahren wir mit der zweiten M�glichkeit ($4 = 2 \cdot 2)$ Also schreiben wir die Rechnung so: \begin{equation*} (2 \cdot 2) \cdot 27 = ?? \end{equation*} Da die Multiplikation kommutativ ist, d�rfen die Faktoren in eine beliebige Reihenfolge gebracht werden. Da die Multiplikation mit 2 einfacher ist als die Multiplikation mit 27 nehmen wir die 27 an den Anfang und multiplizieren diese zweimal mit der Zwei.
\begin{equation*} (27 \cdot 2) \cdot 2 = (54) \cdot 2 = 108 \end{equation*}

\item Es ist jedoch auch erlaubt, die 27 in Faktoren zu zerlegen. Hier gibt es wiederum zwei M�glichkeiten: $1 \cdot 27$ und $3 \cdot 9$. Wie im vorherigen Beispiel wird mit der Umformung von 27 zu $27 \cdot 1$ keine Vereinfachung herbeigef�hrt. Deshalb fahren wir weiter mit der Umformung von 27 zu $3 \cdot 9$ fort und erhalten die folgende Rechnung:
\begin{equation*} 4 \cdot (27) = 4 \cdot (3 \cdot 9) = 4 \cdot 3 \cdot 9 = 12 \cdot 9 \end{equation*}
So wurde aus der Rechnung $4 \cdot 27$ eine v�llig andere Rechnung $ 9 \cdot 12$. Wie man das einfach ausrechnen kann, wird im n�chsten Kapitel erkl�rt.
\end{itemize}

\subsubsection{Summen Bilden}
Um diesen Trick anzuwenden macht man sich das Distributivgesetz zu nutze indem man sich aus einem Faktor eine Summe bildet. Es entstehen zwei Teilrechnungen, deshalb macht es vor allem dann Sinn, diesen Trick anzuwenden, wenn man das eine Teilresultat bereits kennt. Vor allem bei der Multiplikation mit Zahlen die nahe an einem Zehner liegen ist dies ein Mittel um sehr schnell an das Resultat zu gelangen. Im vorangehenden Kapitel tauchte die Rechnung $7 \cdot 17$ auf. 
\textcolor{red}{Vielfach wird das Distributivgesetz unbewusst angewandt: 15 * 7 = 10*7 + 5*7} \\
Man wandelt 17 in eine Summe um, es bietet sich an die Summanden so zu w�hlen, dass die weiteren Rechenschritte m�glichst einfach sind, also w�hlt man in diesem Beispiel die Summanden 10 und 7. Danach multiplizieren wir diese Summe unter Anwendung des Distributivgesetzes aus: \begin{equation*} 7 \cdot (10 + 7) = 10\cdot7 + 7\cdot 7 = 70 + 49 = 119 \end{equation*}
Die Multiplikation mit 10 ist keine schwierige Sache, lediglich eine null muss am Ende der Zahl angeh�ngt werden und die zweite Teilrechnung $(7\cdot7)$ geh�rt zum kleinen Einmaleins und ist den meisten gel�ufig. Also erh�lt man die Zwischenresultate ohne grosse Rechnereien.

\subsubsection{Doppelte Summe}
Die Methode "`doppelte Summe"' ist im Grunde genommen nichts anderes als die Weiterf�hrung der Methode "`Summen bilden"'. Statt eine Zahl in eine Summe umzuwandeln, bildet man in diesem Fall aus beiden Zahlen eine Summe bildet. Allgemein sieht eine Multiplikation so aus:
\begin{equation*} (a+b) \cdot (c+d) = ac + ad + bc + bd \end{equation*}  Diese etwas unsch�ne Formel, kann man vereinfachen, unter Einf�hrung der Nebenbedingung, dass in beiden Summen der gleiche Summand vorkommen soll. Wir gehen davon aus, dass die Summen so gew�hlt wurden, dass die Variabel a und c den gleichen Wert haben und deshalb beide mit der Variabel a dargestellt werden k�nnen:
\begin{equation*} (a+b) \cdot (a+d) = a^2 + ab + ad + bd\end{equation*}
Unter Anwendung des Distributivgesetzes (kap. \ref{distributiv}) kann der Term wie folgt vereinfacht werden:
\begin{equation*} a^2 + ab + ad + bd = a^2 + a\cdot(b+d) + bd \end{equation*}
Somit wurde die Multiplikation von zwei Faktoren umgewandelt in die Addition von einer Quadratzahl und zwei Produkten.
Damit diese Endformel auch wirklich f�r jedermann im Kopf l�sbar ist, sollten die Zahlen geschickt gew�hlt werden. Besonders einfach wird die Rechnung, wenn die Variable a so gew�hlt wird, dass sie eine ganze Zehnerzahl ist. Dass man dank diesem Kniff eine schwierige Multiplikation, im Kopf l�sen kann, m�chte ich mit folgendem Beispiel untermauern.
\begin{equation*} 23 \cdot 16 = ? \end{equation*} als erstes wandeln wir beide Faktoren in eine Summe um und achten darauf, dass in beiden Summen der gleiche Summand vorkommt. Besonders geschickt ist es, diesen Summanden als eine ganze Zehnerzahl zu w�hlen. Hat man zwei Zahlen in der N�he des gleichen Zehners, in unserem Beispiel 20, so macht es sinn Summen mit diesem Zehner zu bilden. Streng genommen ist die Umwandlung von $16$ zu $(20 - 4)$ keine Addition sondern eine Subtraktion, auch das ist erlaubt. \textcolor{green}{Einschub, dass Subtraktion 20 - 4 auch als Addition (20 + (-4)) angesehen werden kann?}
\begin{equation*} (20+3) \cdot (20-4) = ?\end{equation*}
\begin{equation*}  (20+3) \cdot (20-4) = 20^2 + 3 \cdot 20 - 4 \cdot 20 - 3 \cdot 4 \end{equation*}
Aus vier Zwischenresultaten wurden drei und von diesen dreien ist eines ($a^2$) eine Quadratzahl, sollte also den Sch�lern bekannt sein. Also bleiben noch zwei Rechenschritte �brig: $a\cdot(b+d)$ Hier ist es wichtig, dass zuerst die Klammer ausgerechnet und erst anschliessend die Multiplikation durchgef�hrt wird, um den Aufwand m�glichst gering zu halten. Zum Schluss muss einzig noch die Rechnung $b \cdot d$ wirklich ausgerechnet werden. Im normalen Fall sollte dies jedoch mit dem kleinen Einmaleins m�glich sein.

\subsubsection*{Ein Spezialfall}
Die dritte binomische Formel sieht wie folgt aus:
\begin{equation*} (a + b) \cdot (a - b) = a^2 - b^2  \end{equation*}
Diese Formel kann man sich zu nutze Machen, sobald man zwei Zahlen haben, die nahe beieinander liegen, bevorzugt mit einem Zehner in der Mitte.  Beispiel: $22 \cdot 18 = (20 + 2) \cdot (20 - 2) = 400 - 4 = 496$