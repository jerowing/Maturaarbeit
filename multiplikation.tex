\subsection{Multiplikation}
\subsubsection*{Was ist eine Multiplikation}

\subsubsection{Faktoren Aufteilen}
Da die Multiplikation assoziativ ist, darf eine Multiplikation in mehrere Schritte aufgeteilt werden. Beispielsweise darf eine Multiplikation mit 4 als eine Multiplikation mit zwei und nochmals mit zwei angesehen werden. Oder eine Multiplikation mit 5 darf als eine Multiplikation mit 10 und eine anschliessende Division mit 2 (�quivalent zu der Multiplikation mit $\frac{1}{2}$
\begin{equation*} (a \cdot b) \cdot c =  a \cdot (b \cdot c) = a \cdot b \cdot c \end{equation*}

\subsubsection{Summen Bilden}
Um diesen Trick anzuwenden macht man sich das Distributivgesetz zu nutze indem man sich aus einem Faktor eine Summe bildet. Es entstehen zwei Teilrechnungen, deshalb macht es vor allem dann Sinn, diesen Trick anzuwenden, wenn man das eine Teilresultat bereits kennt. \textcolor{red}{Beispiel Einbauen}

\subsubsection{Fingertricks}