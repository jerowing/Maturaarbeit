\subsection{Anwendungen bei der Multiplikation}
Die Multiplikation wird allgemein wie folgt dargestellt und kann aus beliebig vielen Faktoren zusammengesetzt sein.
\begin{equation*} Faktor_1 \cdot Faktor_2 \cdot Faktor_3 \cdot \ldots \cdot Faktor_n = Produkt \end{equation*}

\subsubsection{\label{summenbilden}Summen bilden}
Wandelt man einen Faktor in eine Summe um, so entstehen zwei Teilrechnungen. Es macht vor allem dann Sinn, diesen Trick anzuwenden, wenn man das eine Teilresultat bereits kennt. Beispiele solcher Teilresultate k�nnen Zehnerzahlen, Quadratzahlen aber auch das kleine Einmaleins sein. Vor allem bei der Multiplikation mit Zahlen, die nahe an einem Zehner liegen, ist dies ein Mittel, um sehr schnell zum Resultat zu gelangen. Dieser Trick ist eine Anwendung des Distributivgesetzes (siehe Kapitel \ref{distributiv}). Vielfach wird das Distributivgesetz unbewusst angewandt. Rechnungen wie ($15 \cdot 7$) werden vielfach intuitiv als ($10\cdot 7 + 5 \cdot 7 = 70 + 35 = 105$) gel�st, was genau der Anwendung dieses Tricks entspricht.
\begin{equation*} 13 \cdot 12 = \text{ ?} \end{equation*}
Wir wissen, dass 12 im Quadrat 144 ergibt, deshalb formen wir 13 zu $(12 + 1)$ um.
\begin{equation*} (12 + 1) \cdot 12 = 12^2 + 12 = 144 + 12 = 156\end{equation*} ohne grossen Rechenaufwand sind wir zum Resultat gelangt.

\subsubsection{\label{doppelteSumme}Doppelte Summe}
Die Methode \emph{doppelte Summe} ist im Grunde genommen nichts anderes als die Weiterf�hrung der Methode \emph{Summen bilden}. Statt einen Faktor in eine Summe umzuwandeln, wie beim Trick zuvor, bildet man in diesem Fall aus beiden Faktoren je eine Summe. Allgemein sieht eine Multiplikation zweier Summen so aus:
\begin{equation*} (a+b) \cdot (c+d) = ac + ad + bc + bd \end{equation*}  
Diese Formel mag unsch�n erscheinen, kann und wird jedoch unter Einf�hrung einer Nebenbedingung vereinfacht werden. Die Nebenbedingung lautet: In beiden Summen soll der gleiche Summand vorkommen. Wir gehen davon aus, dass die Summen so gew�hlt wurden, dass die Variablen $a$ und $c$ den gleichen Wert haben und deshalb beide mit der Variable $a$ dargestellt werden k�nnen:
\begin{equation*} (a+b) \cdot (a+d) = a^2 + ab + ad + bd\end{equation*}
Unter Anwendung des Distributivgesetzes (Kap. \ref{distributiv}) kann der Term wie folgt vereinfacht werden:
\begin{equation*} a^2 + ab + ad + bd = a^2 + a\cdot(b+d) + bd \end{equation*}
Somit wurde die Multiplikation von zwei Faktoren umgewandelt in die Addition von einer Quadratzahl und zwei Produkten.
Besonders einfach wird die Rechnung, wenn die Variable $a$ so gew�hlt wird, dass sie eine ganze Zehnerzahl ist.
\begin{equation*} 23 \cdot 16 =\text{ ?} \end{equation*} 
Als erstes wandeln wir beide Faktoren in eine Summe um und achten darauf, dass in beiden Summen der gleiche Summand vorkommt. Befinden sich zwei Faktoren in der N�he des gleichen Zehners, in unserem Beispiel 20, so macht es Sinn, die Summen mit diesem Zehner zu bilden. Streng genommen ist die Umwandlung von $16$ zu $(20 - 4)$ keine Addition sondern eine Subtraktion. Auch das ist erlaubt. \begin{equation*} (20+3) \cdot (20-4) =\text{ ?} \end{equation*}
\begin{equation*}  (20+3) \cdot (20 - 4) = 20^2 + (3 - 4) \cdot 20 - 3 \cdot 4  = \text{ ?} \end{equation*}
Von diesen drei Zwischenresultaten ist eines ($20^2$) eine Quadratzahl, sollte also den Sch�lerInnen bekannt sein. Zwei Rechenschritte m�ssen noch durchgef�hrt werden: $20\cdot(3-4) = 20 \cdot (-1) = -20$. Hier ist es wichtig, dass zuerst die Klammer ausgerechnet und erst anschliessend die Multiplikation durchgef�hrt wird. Damit wird der Aufwand m�glichst gering gehalten. Zum Schluss muss einzig noch die Rechnung $3 \cdot (-4) = -12$ wirklich ausgerechnet werden. Also haben wir zum Schluss die folgende Rechnung:
\begin{equation*} 23 \cdot 16 = 400 - 20 - 12 = 400 - 32 = 368\end{equation*}

\subsubsection*{Ein Spezialfall}
Die dritte binomische Formel sieht wie folgt aus:
\begin{equation*} (a + b) \cdot (a - b) = a^2 - b^2  \end{equation*}
Liegen die Zahlen g�nstig, so kann man die dritte binomische Formel anwenden. Die dritte binomische Formel kann dann angewendet werden, wenn zwei unterschiedliche Zahlen den gleichen Abstand zu einer dritten Zahl haben. Einige Beispiele, um diesen Spezialfall zu verdeutlichen:
\begin{equation*} 22 \cdot 18 = (20 + 2) \cdot (20 - 2) = 20^2 - 2^2 = 400 - 4 = 396 \end{equation*}
\begin{equation*} 14 \cdot 16 = (15 - 1) \cdot (15 + 1) = 15^2 - 1^2 = 225 - 1 = 224 \end{equation*}

\subsubsection{Faktoren aufteilen}
Eine Multiplikation kann in mehrere Schritte aufgeteilt werden. Beispielsweise darf eine Multiplikation mit 4 als eine Multiplikation mit 2 und nochmals mit 2 angesehen werden. Oder eine Multiplikation mit 5 darf als eine Multiplikation mit 10 und eine anschliessende Division mit 2 (�quivalent zu der Multiplikation mit $\frac{10}{2}$) angesehen werden. Grund daf�r ist das Assoziativgesetz (Kapitel \ref{assoziativ}). Die Faktoren einer Multiplikation d�rfen also wiederum in Faktoren zerlegt werden, vorausgesetzt ihr Produkt entspricht dem urspr�nglichen Faktor.
 \begin{equation*} 4 \cdot 27 =\text{ ?} \end{equation*}
Betrachtet man die formale Schreibweise des multiplikativen Assoziativgesetzes (Formel \ref{assoziativ} auf Seite \pageref{assoziativ}), so d�rfen beide Faktoren in Faktoren aufgeteilt werden. Man k�nnte den ersten Faktor ($4$) in Faktoren aufteilen, hierf�r gibt es zwei M�glichkeiten ($1\cdot4$) oder ($2\cdot2$). Nat�rlich g�be es noch weitere Multiplikationen mit dem Ergebnis 4. Diese Rechnungen w�rden jedoch negative Zahlen und / oder Br�che beinhalten. Sowohl negative Zahlen als auch Br�che sind schwieriger zu handhaben als positive, ganze Zahlen. Da mit der Umformung ($4 = 1 \cdot 4$) keine wirkliche Ver�nderung stattfindet, fahren wir mit der zweiten M�glichkeit ($4 = 2 \cdot 2)$ fort. Also schreiben wir die Rechnung so: 
\begin{equation*} (2 \cdot 2) \cdot 27 = \text{ ?} \end{equation*} 
Da die Multiplikation auch kommutativ ist, d�rfen die Faktoren in eine beliebige Reihenfolge gebracht werden. Da die Multiplikation mit 2 einfacher ist als die Multiplikation mit 27, nehmen wir die 27 an den Anfang und multiplizieren diese zweimal mit 2.
\begin{equation*} (27 \cdot 2) \cdot 2 = (54) \cdot 2 = 108 \end{equation*} Es ist jedoch auch erlaubt, die $27$ in Faktoren zu zerlegen. Hier gibt es wiederum verschiedene M�glichkeiten: ($1 \cdot 27$), ($3 \cdot 9$) und ($3\cdot3\cdot3$). Da mit der Umformung von 27 zu ($27 \cdot 1$) keine Vereinfachung herbeigef�hrt wird und wir uns auf zwei Faktoren beschr�nken wollen, fahren wir weiter mit der Umformung von 27 zu ($3 \cdot 9$) und erhalten die folgende Rechnung:
\begin{equation*} 4 \cdot (27) = 4 \cdot (3 \cdot 9) = 4 \cdot 3 \cdot 9 = 12 \cdot 9 = 108 \end{equation*}
So wurde aus der Rechnung $4 \cdot 27$ eine v�llig andere Rechnung $ 9 \cdot 12$. Die Rechnung, die wir am Schluss erhalten haben ($9 \cdot 12$), l�sst sich beispielsweise mit dem Trick \emph{summen bilden} einfach l�sen.
