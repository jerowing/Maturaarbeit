\subsection{Anwendungen bei der Subtraktion}
Bei einer Subtraktion werden eine oder mehrere Zahlen von einer Zahl abgezogen. Der Allgemeine Fall wird so dargestellt.
\begin{equation*} Minuend - Subtrahend_1 - Subtrahend_2 - \ldots - Subtrahend_n = Differenz \end{equation*} Wie bereits bei der Addition werde ich auch die schriftliche Subtraktion nicht weiter erl�utern.

\subsubsection{Erg�nzen auf 1000}
Der folgende Trick zeigt, wie man einfach Zahlen von 1000 subtrahieren kann. Dieser Trick kann auch auf alle anderen Zehnerpotenzen (10, 100, 1'000, 10'000, \ldots) angewendet werden, ich beschr�nke mich in der Arbeit auf die Subtraktion von 1000, da diese am h�ufigsten gebraucht wird im Kopfrechnen. Am Beispiel der Rechnung 1000 - 738 m�chte ich erkl�ren, wie man die Subtraktion von 1000 und vereinfachen kann.In Abbildung \ref{sub1000} ist zu sehen, dass in allen Spalten ausser der Einerstelle ein �bertrag von 1 entstanden, dies ist keine �berraschung, denn zieht man von Null eine Zahl die ungleich 0 ist ab, so muss man sich einen Zehner von der n�chst gr�sseren Stelle borgen. Dies ist bei der Subtraktion von 1000 immer der Fall! Muss man eine Zahl von einer Potenz von zehn abziehen, so kann man alle Stellen auf 9 erg�nzen und die Einerstelle auf 10. Die Regel lautet wie folgt: \textbf{Subtraktion von 1000: Alle von 9 abziehen, die letzte von 10.}

\begin{equation*} 1000 - 738 = \text{ ?} \end{equation*}
Um das Resultat zu erlangen erg�nzen wir also 7 auf 9 (ergibt 2), 3 auf 9 (ergibt 6) und 8 auf 10 (ergibt 2) . Zum Schluss schreiben wir die Zahlen in dieser Reihenfolge nieder: \begin{equation*} 1000 - 738 = 262\end{equation*}
\begin{figure}[!h]
\label{SubtraktionRundeZahl}
\centering
\begin{tabular}{r r r r r}
&Tausender & Hunderter & Zehner & Einer \\ \cline{1-5}
&1 & 0 & 0 & 0 \\
-&  & 7 & 3 & 8 \\
�bertr�ge & 1 &1 & 1 &  \\ \cline{2-5}
&&  2& 6 & 2 \\ \hline \hline
\end{tabular}
\caption[Eigene Darstellung]{\label{sub1000}Die Subtraktion von 738 von 1000}
\end{figure}


\subsubsection{Der unsichtbare Helfer} \label{unsichtbar}
Die Differenz gibt an, wie gross der Unterschied zwischen zwei Zahlen ist. Als Beispiel m�chte ich zwei fiktive nebeneinanderstehende Hochh�user ins Leben rufen, das eine ist 10 Meter h�her als das andere. Meine �berlegungen zu diesem Beispiel: Niemand hat definiert, von wo aus der H�henunterschied zwischen den beiden H�usern korrekterweise gemessen werden muss. Man k�nnte von Dach des einen Hochhauses messen, wie auch von der Oberfl�che, auf der die beiden H�user stehen. Die Distanz bleibt die gleiche, selbst wenn aus dem Weltraum gemessen w�rde. Aus dieser �berlegung habe ich den folgenden Trick abgeleitet: die Differenz ver�ndert sich nicht, wenn sowohl beim Minuenden als auch beim Subtrahenden die gleiche Zahl addiert oder subtrahiert wird. Die algebraische Begr�ndung sieht so aus: 
\begin{equation*} (a - x) - (b - x) =  a - x - b + x = a - b - x + x = a - b \end{equation*}
\begin{equation*} (a + x) - (b + x) =  a + x - b - x = a - b  + x - x = a - b\end{equation*}
Weil das Minus die Vorzeichen in der Klammer wechselt, entstehen im Zusammenhang mit x zwei gegenteilige Vorzeichen, die sich gegenseitig ausl�schen.

Es ist sinnvoll, entweder den Minuenden oder den Subtrahenden auf eine runde Zahl zu erg�nzen, denn so l�sst es sich besonders einfach weiterrechnen. Wie die Methode der unsichtbare Helfer funktioniert m�chte ich am folgenden Beispiel verdeutlichen:
\begin{equation*} 285 - 93 = \text{ ?} \end{equation*}
Wir erg�nzen also, naheliegender Weise den Subtrahenden auf die n�chste runde Zahl, in diesem Fall 100. Damit sich die Differenz nicht ver�ndert, m�ssen wir jedoch auch den Minuenden auf die gleiche Weise ver�ndern, folglich z�hlen wir sowohl zum Minuenden als auch zum Subtrahenden 7 dazu. Daraus entsteht die folgende Rechnung:
\begin{equation*} 285 - 93 = (285 + 7) - (93 + 7) = 292 - 100 = 192\end{equation*}
Nat�rlich gibt es auch hier eine zweite M�glichkeit die Rechnung zu vereinfachen indem man 285 auf eine Zahl erg�nzt, mit der es sich einfach rechnen l�sst (beispielsweise 300) und bei 93 gleichviel dazuz�hlt:
\begin{equation*} 285 - 93 =  (285 + 15) - (93 + 15) = 300 - 108 = 192\end{equation*}
Jedoch bevorzuge ich den ersten Weg, da der Rechenaufwand geringer ist, einen ganzen Hunderter abzuziehen anstatt von einem ganzen Hunderter abzuziehen.

\subsubsection{Kettenrechnungen}
Zieht man nacheinander mehrere Zahlen von einer Zahl ab, kann die Summe aller Subtrahenden vom Minuenden abgezogen werden. Dies l�sst sich ganz einfach mit dem Trick bewirken, das man alle Subtrahenden in eine Klammer setzt und ein Minus davor. Denn das Minus wechselt alle Vorzeichen in der Klammer.
\begin{equation*} \label{kettenrechnungsubtraktion} a - b - c - d = a - (b + c + d) \end{equation*} 
So entstehen zwei Teilrechnungen zum einen die Addition der Summanden b, c und d zum andern die Subtraktion dieser Summe vom Minuenden a. Bei der Addition von b, c und d handelt es sich um eine Kettenrechnung, welche man mit Hilfe des Tricks \emph{Gruppen bilden} (Kapitel \ref{gruppen}) l�sen kann. Die anschliessende Subtraktion kann mit Hilfe der zuvor erkl�rten Tricks f�r die Subtraktion im Kopf gel�st werden.
\begin{equation*} 1000 - 232 - 137 - 288 = \text{ ?}  \end{equation*}
Zuerst fassen wir alle Minuenden als Summe in einer Klammer zusammen und stellen die Summanden in der Klammer so um, damit sie sich einfach zusammenz�hlen lassen:
\begin{equation*}1000 - (232 + 173 + 288) = 1000 - (232 + 288 + 173) = \text{ ?}\end{equation*}
Anschliessend rechnen wir die Klammer aus und subtrahieren die ergebene Summe vom Minuenden:
\begin{equation*} 1000 - (520 + 173) = 1000 - 693 = 307\end{equation*}