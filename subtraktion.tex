\subsection{Subtraktion}
Bei einer Subtraktion werden eine oder mehrere Zahlen von einer Zahl abgezogen. Der Allgemeine Fall wird so dargestellt.
\begin{equation} Minuend - Subtrahend_1 - Subtrahend_2 - \ldots - Subtrahend_n = Differenz \end{equation}
\subsubsection{die schriftliche Subtraktion}
Wie bereits bei der Addition werde ich auch die schriftliche Subtraktion nur aus dem Gr�nden der Anschaulichkeit hier besprechen. Wie bei der schriftlichen Addition werden die Zahlen untereinander hingeschrieben so dass Einer-, Zehner- und Hunderterstelle �bereinander stehen. Danach wird in jeder Spalte einzeln jeweils alle Subtrahenden vom Minuenden abgezogen. Ist die Summe der Subtrahenden kleiner als der Minuend, so kann die Differenz problemlos ausgerechnet werden und in der jeweiligen Spalte als Ergebnis eingetragen werden. Etwas komplizierter wird es, wenn die Summe der Subtrahenden gr�sser ist als der Minuend, also eine grosse Zahl von einer kleinen abgezogen werden soll. Da wir uns in der Menge der nat�rlichen Zahlen bewegen gibt es keine negativen Zahlen und Resultate. Deshalb muss man, damit die Subtraktion dennoch vollf�hrt werden kann von der n�chsth�heren Stelle einen oder mehrere Zehner geborgt werden. (in der Abbildung gr�n hervorgehoben.) Wer m�chte kann es auch wie folgt umstellen und den �berschlag als zus�tzlichen Subtrahenden in der jeweiligen Zeile hineinschreiben. Eine alternative Methode findet sich im Kapitel \ref{Kettenrechnungen}. \textcolor{red}{Summe bilden und erg�nzen im Kapitel Gruppen bilden erkl�ren!}
\begin{figure}[!h]
\label{schriftlicheSubtraktion}
\centering
\begin{tabular}{r r r r r}
&& Hunderter & Zehner & Einer \\ \cline{1-5}
Minuend && (6 \textcolor{green}{-1}) & 2 & 8 \\
Subtrahend & - & 1 & 8 & 2 \\ \cline{2-5}
Differenz&& 4 &4 &6 \\ \hline \hline
 
 \end{tabular}
 \caption{Die schriftliche Subtraktion: 628 - 182}
\end{figure}


\subsubsection{Kettenrechnungen}
Zieht man nacheinander mehrere Zahlen von einer Zahl ab, kann die Summe aller Subtrahenden vom Minuenden abgezogen werden. Dies l�sst sich ganz einfach mit dem Trick beweisen, das man alle Subtrahenden in eine Klammer setzt und ein Minus davor. Denn das Minus wechselt alle Vorzeichen in der Klammer.
\begin{equation*} a - b - c - d = a - (b + c + d) \end{equation*}
\begin{figure}[!h]
\label{schriftlicheSubtraktion2}
\centering
\begin{tabular}{r r r r r}
&& Hunderter & Zehner & Einer \\ \cline{1-5}
Minuend&& (5 \textcolor{green}{- 1}) & 2 & 3 \\
Subtrahend$_1$& - & 3 & 4 & 1 \\
Subtrahend$_2$& - & 1 & 3 & 1 \\ \cline{2-5}
Differenz &&  & 5 & 1 \\ \hline \hline
 
 \end{tabular}
 \caption{Die schriftliche Subtraktion: 523 - 341 - 131}
\end{figure}

\subsubsection{[mein Trick] der unsichtbare Helfer}
Die Differenz gibt an, wie gross der Unterschied zwischen zwei Zahlen ist. Als Beispiel m�chte ich zwei fiktive nebeneinanderstehende Hochh�user ins Leben rufen, das eine ist 10 Meter h�her als das andere. Meine �berlegungen zu diesem Beispiel: Ich rufe drei Betrachter ins Leben: der erste Betrachter ist auf der H�he des zehnten Stockwerkes der Hochh�user, der zweite auf der H�he des Erdgeschosses und der dritte Betrachter befindet sich im Erdmittelpunkt. Nun wollen alle drei Betrachter herausfinden wie gross der H�henunterschied zwischen den beiden H�usern ist. Erhalten sie verschiedene Resultate? Nein erhalten sie nicht. Aus dieser �berlegung habe ich den folgenden Trick abgeleitet: die Differenz ver�ndert sich nicht, wenn sowohl beim Minuenden als auch beim Subtrahenden die gleiche Zahl addiert oder subtrahiert wird. Der algebraische Beweis sieht so aus: 
\begin{equation*} (a - x) - (b - x) =  a - x - b + x = a - b - x + x = a - b \end{equation*}
\begin{equation*} (a + x) - (b + x) =  a + x - b - x = a - b  + x - x = a - b\end{equation*}
Weil das Minus die Vorzeichen in der Klammer wechselt, entstehen im Zusammenhang mit x zwei gegenteilige Vorzeichen, die sich gegenseitig ausl�schen.
\subsubsection{Subtraktion einer sch�nen Zahl}
Der Begriff sch�ne Zahlen wurde in der Einleitung definiert. Am Beispiel der Rechnung 1000 - 738 m�chte ich erkl�ren, wie man die Subtraktion von einer Runden zahl vereinfachen kann:
\begin{figure}[!h]
\label{SubtraktionRundeZahl}
\centering
\begin{tabular}{r r r r r}
&Tausender & Hunderter & Zehner & Einer \\ \cline{1-5}
&1 & 0 & 0 & 0 \\
-&  & 7 & 3 & 8 \\
�bertr�ge & \textcolor{green}{1} &\textcolor{green}{1} & \textcolor{green}{1} &  \\ \cline{2-5}
&&  2& 6 & 2 \\ \hline \hline
 \end{tabular}
 \caption{Die Subtraktion von einer sch�nen Zahl}
\end{figure}
Wie man sieht ist in allen Spalten ausser der Einerstelle ein �bertrag von 1 entstanden, dies ist keine �berraschung, denn zieht man von Null eine Zahl die ungleich 0 ist ab, so muss man sich einen Zehner von der n�chst gr�sseren Stelle borgen. Da das f�r alle sch�nen Zahlen der Fall ist, kann man die folgende Regel ableiten: \textbf{Subtraktion einer sch�nen Zahl: Alle von 9 abziehen, die letzte von 10.}

%\subsubsection{R�dchenmethode}
%Wie bereits bei der Addition kann auch die Subtraktion mit einem Rad verbildlicht werden. Eine Abbildung eines solchen Rades ist auf Seite \pageref{rad} zu finden. Das System funktioniert auf die gleiche Weise wie bei der Addition nur dass die Umlaufrichtung im gegenuhrzeigersinn gew�hlt werden muss. Auch der Zehner�berschlag beim �berschreiten der Null bleibt exakt gleich, nur dass er von der n�chsten Stelle abgezogen werden muss statt hinzuzuz�hlen.

